\documentclass[10pt,twocolumn,letterpaper]{article}

\usepackage{cvpr}
\usepackage{times}
\usepackage{epsfig}
\usepackage{graphicx}
\usepackage{amsmath}
\usepackage{amssymb}
\usepackage{multirow}

% Include other packages here, before hyperref.

% If you comment hyperref and then uncomment it, you should delete
% egpaper.aux before re-running latex.  (Or just hit 'q' on the first latex
% run, let it finish, and you should be clear).
\usepackage[breaklinks=true,bookmarks=false]{hyperref}

\cvprfinalcopy % *** Uncomment this line for the final submission

\def\cvprPaperID{****} % *** Enter the CVPR Paper ID here
\def\httilde{\mbox{\tt\raisebox{-.5ex}{\symbol{126}}}}

% Pages are numbered in submission mode, and unnumbered in camera-ready
\ifcvprfinal\pagestyle{empty}\fi
%\setcounter{page}{4321}
\begin{document}

%%%%%%%%% TITLE
\title{A convolutional classification approach to colorization}

\author{Vincent Billaut\\
Department of Statistics\\
Stanford University\\
{\tt\small vbillaut@stanford.edu}
% For a paper whose authors are all at the same institution,
% omit the following lines up until the closing ``}''.
% Additional authors and addresses can be added with ``\and'',
% just like the second author.
% To save space, use either the email address or home page, not both
\and
Matthieu de Rochemonteix\\
Department of Statistics\\
Stanford University\\
{\tt\small mderoche@stanford.edu}
\and
Marc Thibault\\
ICME\\
Stanford University\\
{\tt\small marcthib@stanford.edu}
}

\maketitle
%\thispagestyle{empty}

%%%%%%%%% ABSTRACT

% TODO for final report

\begin{abstract}
Insert abstract here
\end{abstract}

%%%%%%%%% BODY TEXT
\section*{Introduction}

The problem of colorization is one that comes quickly to mind when thinking about interesting challenges involving pictural data. Namely, the goal is to build a model that takes the greyscale version of an image (or even an actual ``black and white'' picture) and outputs its colorized version, as close to the original as possible (or at least realistic, if the original is not in colors).
One clear upside to this challenge is that any computer vision dataset, and even any image bank really, is a proper dataset for the colorization problem (the image itself is the model's expected output, and its greyscale version is the input to the model).

Classical approaches to this task, \eg \cite{cheng2015deep} and \cite{dahl2016tinyclouds}, aim at predicting an image as close as possible to the ground truth, and notably make use of a simple $L_2$ loss, which penalizes predictions that fall overall too far from the ground truth. As a consequence, the models trained following such methods usually tend to be very conservative, and to give desaturated, pale results.
On the contrary, authors of \cite{zhang2016colorful} take another angle and set their objective to be ``\textit{plausible} colorization'' (and not necessarily \textit{accurate}), which they validate with a large-scale human trial.

Our goal is to reproduce the approach of \cite{zhang2016colorful}, as we consider the implementation more challenging in the way the loss function and the prediction mechanism are designed, and the results more visually appealing.

If we reach satisfactory results in a timely manner, we will consider tackling the task of colorizing videos, for which, in order to fully take advantage of the input format, will need to incorporate the notion of consistency between consecutive frames in a sequence.

\section{Related Work}
\section{Methods}
\section{Dataset and Features}
\section{Results and discussion}
\section{Conclusion and perspectives}
\section*{Contributions and acknowledgements}


{\small
\bibliographystyle{ieee}
\bibliography{references}
}

\end{document}
